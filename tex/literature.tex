\section{Related Works}
\label{chp:relworks}
Compared to data mining, stream mining is particularly a very young area. Most of the methods date back only couple of decades. Similarly, concept of the ensemble learning was first introduced in the traditional batched learning, however, ensemble adaptations for streams are fairly newer concepts. In this chapter, we list the most influential methods developed so far with particular focus on tree based learners. We follow a rudimentary narrating style starting directly with methods for general stream learners for stationary streams, and then for evolving data. After that, we list the base methods for ensemble learning, and finally present ensemble based stream learners.

\subsection{Stream Mining}
%Compared to the classical data mining approaches stream mining is relatively a newer topic to be addressed in literature. Even though for batched approaches both classification and clustering problems have been vastly studied, their stream adaption remains a challenge due the restrictions imposed by the stream data. Possibility of temporal locality makes the classification problem harder in a streaming environment. Algorithms needs to address the evolution of underlying data stream. 

Domingos and Hulten introduced a strict one-pass adaptation of decision tree~\cite{breiman84:dt,quinlan93:c45} approach in streams. Classic approaches like ID3 and C4.5 learners assume that all training examples can be stored in the main memory altogether. This is a significant limitation to the number of examples these algorithms can handle. Similarly, disk based decision tree learners (SLIQ~\cite{mehta96:sliq}, SPRINT~\cite{shafer96:sprint}, etc.) become very expensive when data sets are very large and the expected trees has many levels. Domingos and Hulten proposed \textit{Very Fast Decision Trees (VFDT)}~\cite{domingos00:vfdt} that uses Hoeffding bound~\cite{hoeffding63:bound} to build an anytime decision tree for constant memory and time. The primary assumption in this approach is that to find the best attribute for a node in a decision tree, it may be sufficient to consider only a fraction of the training set that passes through that node. Hoeffding bound provides a statistical measure to determine how much data is needed to ensure a level of degree of certainty, i.e. error margin would be bounded by a given value~\cite{catlett91:thesis}.

Like most statistical and machine leaning algorithms VFDT assumes that training data is randomly drawn from a stationary distribution. This assumption is not valid for large databases and data streams. Over time underlying method or environment could change that generates data. The shift is  referred as {\it concept drift} in literature and can be abrupt as well as very slow. Data related to weather forecast, economic condition prediction, mis-calibrated sensors, etc. are examples of concept drifting environment. A concept-adaptive variant of VFDT, \textit{CVFDT}~\cite{hulten01:cvfdt}, can handle such scenarios. CVFDT updates its decision rules, essentially the tree structure, by detecting the concept drift in the data. It maintains alternate subtrees whenever an old subtree becomes questionable, and replaces the old one with the alternative when it becomes more accurate. CVFDT uses a sliding window and updates sufficient statistics by increasing the count of newly arrived examples and decreasing the count of old examples in the window. Essentially CVFDT achieves same accuracy that would be achieved if VFDT had been run again with the new data. CVFDT does this in $O(1)$ with additional space requirement as compared to the VFDT's $O(w)$ where $w$ is the window size. Another extension of VFDT, VFDTc was proposed in~\cite{gama05:vfdtc} that improves VFDT by handling continuous numeric attribute  and adding naive Bayes prediction at the leaves.


% TODO: briefly include multi-label evolving ...
% TODO: frequent item set mining]

\subsection{Ensemble Learning}
Traditional machine learning algorithms generally feature a single model or classifier such as \textit{Na\"ive Bayes} or \textit{Multilayer Perceptron (MLP)}. The free parameters of these learners (e.g. weights of feed-forward neural network) are set by realizing the complete training set. These classifiers provide a measurement of the generalization performance i.e. how well the classifier generalizes the training set. However, given a finite set of training example, it is rather reasonable to assume that the data might contain several generalizations. For example, a different setting of neural network classifier (weights, node layers, node counts, etc.) changes the final network to some extent. For stream environment, this assumption becomes primitive. Thus, choosing a single classifier is not always optimal. Using the best classifier among several classifiers where each are trained with same training set would be an alternative, however, information is still being lost by discarding sub-optimal options. A better alternative would be to build a classifier ensemble. Ensemble classifiers combine the prediction of multiple base level model built on traditional algorithms. A simple process for combining prediction could be to choose the decision based on majority voting~\cite{parhami94:voting}. As demonstrated in several works~\cite{breiman93:regression, schapire90:whyens, wolpert92:whyens}, ensemble methods (e.g. ensembles of neural networks)~\cite{hansen90:ensNN, tumer99:whyens} yield better performance. 

Without proper selection and control over the training process of the base learners, ensemble classifiers could result in poorer performance. Simply choosing a base classifier and training it for several settings would surely produce highly correlated classifiers which would have adverse effect on the overall ensemble process. One solution of this  is to train each classifier with its own training set generated by random sampling the original one. However, with random sampling each classifier would receive a reduced number of training patterns, resulting a reduction in the accuracy of the individual base classifier. This reduction in the base classifier accuracy is generally not recovered by the gain of combining the classifier unless measures are taken to make the base classifiers \textit{diverse}. Classifiers with complementary information would give the lowest correlation~\cite{breiman93:regression, tumer99:whyens}. Many methods have been proposed to promote diversity among the base classifier: \textit{bagging}~\cite{breiman94:bagging}, \textit{boosting}~\cite{drucker94:boosting, freund97:boosting, oza99:whyens}, \textit{cross-validation partitioning}~\cite{krogh95:ensNNcv, tumer99:whyens}, etc. These methods mainly process the entire training set repeatedly and require at least one pass for each base model. This is not suitable for streaming scenarios. Stream adaptation of bagging and boosting methods has been introduced by Oza et al.~\cite{oza01:obagboost,oza01:thesis}.

\subsection{Ensemble Learning in Streams}
Learning algorithms in data streams require maintenance of a set of hypotheses based on the training instances seen thus far with the need for storage and reprocessing. Facilitating this requirement, Oza and Russell developed an online version~\cite{oza01:obagboost, oza01:thesis} of traditional bagging and boosting. Bagging works by randomly sampling with replacement from the training set to form a given number of intermediate training sets which are used to train same number of classifiers. During testing a majority voting scheme is employed on the decisions of all classifiers to deduce the final decision. Boosting uses an iterative procedure to adaptively change distribution of training data by focusing more precisely on misclassified instances. Initially all instances have equal weights, and at the end of a boosting round weight of each instance is updated by increasing or decreasing if the instance was classified wrongly or correctly, respectively. For \textit{online} variant of these algorithms, not knowing the size of the training data poses a problem in determining the size of training sets to build the base models. In~\cite{oza01:obagboost} authors address this situation by training $k$ models with each instances where $k$ is a suitable Poisson random variable. Later on,~\cite{pelossof08:boosting} proposed the \textit{Online Coordinate Boosting} algorithm where the number of weight updates of~\cite{oza01:obagboost} is reduced using few simple alteration.


As mentioned in the previous section, \textit{Hoeffding Tree (HT)} i.e. VFDT~\cite{domingos00:vfdt}, can be used to build classifiers for concept drifting streams. The Hoeffding Tree has the property that it adapts itself for the newer examples. The number of examples that a HT is build upon is determined by two numbers: (i) the  size of the tree, and (ii) the number of examples used to create a node. Thus, smaller trees adapt faster to the changes in the data, while larger trees try to retain the rules that reflect longer time-frame, simply because they are built on more data. In other words tree bounded by size $n$ would be reset twice as often as tree bounded by size $2n$. \textit{Adaptive Size Hoeffding Tree (ASHT)} bagging~\cite{bifet09:asht}, uses this intuition to build an ensemble of classifiers of different sized Hoeffding trees. ASHT bagging attempts to increase the diversity in the bagging approach. The maximum allowed size for $n$-th tree is twice the size of $(n-1)$-th, where the 1st tree has a size of $2$. Additionally, inverse of the squared error has been used as the weights for the trees. Authors made an observation that bagging using 5 trees of different size might be sufficient for gain higher accuracy, as error level for bagging with more trees do not improve much but takes longer time.

Same authors also proposed an adaptive window size bagging method- \textit{ADWIN}~\cite{bifet09:asht}. ADWIN automatically detects and adapts to the current rate of change. To do so ADWIN adapts its window size to maximize the statistically consistently length that conforms following hypothesis ``there has been no change in the average value inside the window''. Window is not maintained explicitly, rather using a variant of exponential histogram technique that takes $O(log w)$ memory and $O(log w)$ processing time where $w$ is the length of the window. Experimental evaluation showed that ADWIN baggin has better accuracy than ASHT bagging, however, requires more time and memory. 

ADWIN has later been used in \textit{leverage bagging}~\cite{bifet10:levbag}. Leverage bagging improves randomization by increasing re-sampling and using output detection codes. Re-sampling with replacement is done in online Bagging using Poisson(1). Instead leverage bagging increases the weights of re-sampling using a larger value $\lambda$ to compute the value of the Poisson distribution. The Poisson distribution is used to model the number of events occurring within a given time interval. In other improvement randomization is added at the output of the ensemble using output codes. Method works by assigning a binary string of length $n$ to each class and building an ensemble of $n$ binary classifiers. Each of the classifiers learns one bit for each position in the string. A new instance is classified to the class whose binary code is closest. 

ADWIN has also been used in building ensemble of \textit{Restricted Hoeffding Trees}~\cite{bifet10:rht}. A mechanism for setting the learning rate of perceptrons using ADWIN's change detection method is used to restrict the tree. Additionally, a mechanism for reseting the member Hoeffding trees is also been introduced when a particular member is no longer performing well. The method outperforms traditional bagging in terms of accuracy, but requires additional memory and time.

