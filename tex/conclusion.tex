\chapter{Conclusion}
\label{chp:conclude}

In this thesis, we have death with a well known problem of supervised classification of data instances in streaming environment. Particularly, we have wanted to observe the behavior of the current decision tree based state-of-the-art approaches for data streams that are composed of irregular sub-streams depending on the number of instances. The motivation was that often a large volume of data is produced by only few sources within a stream and they dominate the classifier's decision rules. This is specially problematic for the sources that produces few instances. We have analyzed common stream generators used for testing stream learners. It was found that most of these generators generate data in a randomize manner. Thus, we have devised a new stream generator where volume of data produced by source is somewhat related to its lifespan (time it remain active). 

We then have tested current methods for both regular and our new generators. The theoretical aspects are supported with the evidence from a large set of experimental data. It was found that some of the existing methods, Hoeffding tree, adaptive Hoeffding tree, and their boost variants, even though performs well, but leads to over-fitting. This problem is solved by using ensemble approach with a maximum size limitations in bagging with adaptive size Hoeffding tree. This method, however, is found to have a limitation. Every time a larger tree is being reset in the ensemble, classification accuracy for instances immediately following drops significantly. 

With this observation, we developed a new bagging approach, which in principle works similarly to the bagging with ASHT approach, to solve the issue. In the new approach we have delayed the reset process while starting to learn a new classifier. We have also used ADWIN change detection approach within each model. In ASHT bagging approach ADWIN is not used. With these few changes, we have managed to achieve smoother performance curves.

In summary, the major contributions of this thesis are follows:

\begin{itemize}
    \item A new look at the composition of various real-world streams e.g. social networks. Based on that a new stream generation approach.
    
    \item A new algorithm, carry-over bagging with SRHT, to solve the limitation of ASHT bagging approach.
    
    \item An extensive empirical study to compare the discussed methods.
    
    \item A comprehensive survey of the existing literature.
\end{itemize}

\section{Future Works \& Open Issues}
The motivation of this thesis was taken from all text based examples. However, to keep the thesis within the scope, the experimental evaluations were performed using numeric datasets. We would be like to extend this work and apply it directly in real world data streams. 

In our current solution, we eventually used every instances to learn the model. An alternate option would be to balance the model by only training though selective instances. The selection process would try to balance among data sources. Thus, learning from a small subset of faster sources and using most instance from slower sources. We would like to investigate this approach in future.





