\chapter{Conclusion}
\label{chp:conclude}

In this thesis, we have dealt with a well-known problem of supervised classification of data instances in a streaming environment. Particularly, we wanted to observe the behavior of the current decision tree based state-of-the-art approaches for data streams that are composed of irregular sub-streams depending on the number of instances. The motivation was that often a large volume of data is produced by only few sources within a stream and they dominate the classifier's decision rules. This is especially problematic for the sources that produce few instances. We have analyzed common stream generators which are used for testing stream learners. It was found that most of these generators generate data in a randomized manner. Thus, we have devised a new stream generator where the volume of data produced by the source is  related to its lifespan (time it remains active). 

We then have tested current methods for both currently available and our new generators. The theoretical aspects are supported with the evidence from a large set of experimental data. It was found that some of the existing methods, e.g. Hoeffding tree, adaptive Hoeffding tree, and their boost variants, perform well, but lead to over-fitting. This problem is solved by using an ensemble approach with a maximum size limitation in bagging with adaptive size Hoeffding tree. This method, however, is found to have a limitation. Every time a larger tree is being reset in the ensemble, the classification accuracy for instances following immediately drops significantly. 

With these observations, we developed a new bagging approach  to solve the issue, which in principle works similarly to the bagging with ASHT approach. In the new approach we have delayed the reset process while starting to learn a new classifier. We have also used the ADWIN change detection approach within each model. In the ASHT bagging approach ADWIN is not used. With these few changes, we have managed to achieve smoother performance curves.

In summary, the major contributions of this thesis are as follows:

\begin{itemize}
    \item A new look at the composition of various real-world streams e.g. social networks. Based on that we developed a new stream generation approach.
    
    \item A new algorithm, carry-over bagging with SRHT, to solve the limitations of ASHT bagging approach.
    
    \item An extensive empirical study to compare the discussed methods.
    
    \item A comprehensive survey of the existing literatures.
\end{itemize}

\section{Future Works \& Open Issues}
The motivation of this thesis was taken from all text based examples. However, to keep the thesis within the scope, the experimental evaluations were performed using numeric data sets. We would like to extend this work and apply it  in real world data streams. 

In our current solution, we eventually used every instances to learn the model. An alternate option would be to balance the model by only training through a set of well-selected instances. The selection process would try to balance among data sources. That means, learning from a small subset of faster sources and using most instances from slower sources. We would like to investigate this approach in the future.





