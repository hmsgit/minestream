\chapter{\abstractname}
To handle the enormous amount of data being produced by various applications nowadays, stream mining algorithms have been developed where the assumption, made in batched approaches, of data being sampled from a stationary distribution is relaxed.
Most of these stream mining approaches assume that sources of these data within the stream contribute evenly to the stream.
However, many of the modern applications violate this assumption. Many streams are decomposable in to a number of sub-streams with mutually exclusive sources. Some of these sources produce very high volumes of data in a short period of time, while others produce a significant amount of data over longer periods. Each source may produce data for all or some classes.
As the learners try to update themselves with the most recent data, they often loose information about these slow but important sources.
In this thesis, we investigate various aspects of such a stream uses Hoeffding tree based learners. We also extend existing algorithms to introduce our Size Restricted Hoeffding Tree (SRHT) and Carry-over Bagging (CoBag) algorithms. Carry-over bagging uses an ensemble of different sized SRHTs to improve classification performances for the slower data streams. With extensive experimentations, we show that the new approach is more stable for learning the data streams that can be decomposed into varying speed sub-streams i.e. sub-streams with different speeds.

