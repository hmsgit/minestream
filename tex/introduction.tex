\chapter{Introduction}
\label{chp:intro}
Huge amount of data are generated in an unprecedented rate now-a-days from different application domains like social networks, telecommunications, WWW, scientific experiments, e-commerce systems, health care systems, etc. These data flow in and out of systems continuously with varying update rates. Banking systems keep registering each of their ATM and account transactions, telecommunication providers log each of their call information, popular websites maintain their user logging details, organization like CERN produces petabytes of data during their experiments; these are known as stream data. The volume and the rate of incoming data make it nearly impossible to mine these data with traditional data mining approaches. As alternatives, mining a subsample of available data or to mine for models much simpler than what the data might support can be performed. This, however, drastically limits the ability to extract information from the data. Moreover, in some cases, accumulating and storing the data in runtime and performing a sampling to apply mining algorithms simultaneously is a challenge. For such reasons the notion of mining fixed-size database is slowly being replaced with the idea of open-ended data streams.

Compared to the classical data mining approaches stream mining is relatively a newer topic to be addressed in literature. Even though for batched approaches both classification and clustering problems have been vastly studied, their stream adaptations remain a challenge due the restrictions imposed by the stream data. Due to the volume and the nature of the stream data there are several restrictions that are needed to be considered while designing a stream mining algorithm. Most important limitations include- not being able to store the complete data set in memory and hence only being able to use an example once to train the model; evolving nature of the data with time changes, etc. Thus stream mining requires different approach than the traditional batch learning. Possibility of temporal locality makes the classification problem even harder in a streaming environment. Algorithms need to address the evolution of underlying data stream.

Tradition machine learning algorithms generally feature a single model or classifier such as Na\"ive Bayes or multilayer perceptron (MLP) learned from the training set and use it to classify testing set. The free parameters of these learners (e.g. weights of feed-forward neural network) are set by realizing the complete training set. These classifiers provide a measurement of the generalized performance i.e. how well the classifiers generalize the training set. Some of the assumptions these algorithms make are: (i) data are finite, (ii) underlying data regularities are stationary, (iii) stationary data sources generate independent and identically distributed data, (iv) data are invariant to the spatial or temporal changes, etc. None of these assumptions is valid for data streams. Stream data exhibit following characteristics: 
\begin{itemize}
    \item Data come as a continuous flow of unlimited stream, often at a very high speed.
    \item Underlying models in the data evolve over time.
    \item Data cannot be considered to be independent and identically distributed.
    \item Time and space can significantly influence data.
\end{itemize}
At the first glance, it may seem that some simple adaptation of current methods should be sufficient to address these changed conditions in data. In reality, these changes challenge most trivial learning methods of machine learning. For example, the computation of entropy of data. For batched learning all the instances and their corresponding classes are known before-hand to compute entropy. However, for streaming case, data stream is no longer finite, number of classes are not known a priori, domain of variables are not necessarily small in size; and hence computation of batched-like entropy function is not possible. Similarly, maintaining a frequent item set in a hundred of gigabytes of database also cannot be easily be derived from traditional machine learning algorithms.

Typically, adaptations of traditional algorithms need to address continuous flow of data, dynamic environment of generating sources, unavailability of complete data set, etc. Following are some of the new requirements that are needed to be considered while developing a knowledge discovery system for stream data:
\begin{itemize}
    \item Algorithms should use limited resources in terms of power, memory, and processing time.
    \item Algorithms should only access the data a limited number of times, and may only use a limited bandwidth.
    \item Algorithms should be ready to predict {\it anytime}. 
    \item Data gathering and processing might be distributed.
\end{itemize}

With traditional algorithms, a single model is learned, i.e. only one single generalization of the data is learned. However, given a finite set of training example, it is rather reasonable to assume that the data might contain several generalizations. For example, a different setting of neural network classifier (weights, node layers, node counts, etc.) might change the final network to some extent. For stream environment this assumption becomes primitive. Thus, choosing a single classifier is not always optimal. Using the best classifier among several classifiers where each is trained with same training set would be an alternative, however, information are still being lost by discarding sub-optimal options. A better alternative would be to build a classifier ensemble. Ensemble classifiers combine the predictions of multiple base level models built on traditional algorithms. A simple process for combining prediction would be to choose the decision based on majority voting. As demonstrated in several works~\cite{breiman94:bagging, schapire90:whyens} such ensemble methods (e.g. ensembles of neural networks) yield better performance.

Without proper selection and control over the training process of the base learners, ensemble classifiers could result in poorer performance. Simply choosing a base classifier and training it for several settings would surely produce highly correlated classifiers which would have adverse effect on the ensemble process. One solution of this issue is to train each classifier with its own training set generated by random sampling over the original one. However, with random sampling each classifier would receive a reduced number of training patterns, resulting in a reduction in the accuracy of the individual base classifiers. This reduction in the base classifiers' accuracy is generally not recovered by the gain of combining the classifiers unless measures are taken to make the base classifiers diverse.

Diversity is generally achieved by making the base classifiers minimally correlated. In recent years, various methods have been developed to address this issue. Among others bagging, boosting, Hoeffding Tree based approaches are mentionable. Some of these approaches are suitable for label based classification, and some can be used to approximate the trend of data over a specific time granularity. It will, however, be nice if a method can approximate the trend of data over a set of time granularity. Moreover, investigating how ensemble methods perform where base classifiers are trained to predict one specific labels should be interesting. For example, given a twitter stream an ensemble of classifiers that may contain base classifiers to classify sentiments for sports, politics, entertainment, etc. separately. It is expected this ensemble setting would out-perform the generic ensemble approach for complete stream altogether.

Ensemble methods build model outputs where abstract properties of the algorithms that produces the model are prioritized rather than the specifics of the algorithms. This allows a wide application across many fields of study. With precise use of ensemble methods, it would be possible to automatically exploit the strengths and weaknesses of different machine learning systems.

This thesis investigates currently available such ensemble approaches, and presents an empirical analysis of those methods. Moreover, this thesis presents a unique perspective in relation to the underlying setup that generates the stream, which applies to certain application domains such as social media. In following sections, we present this motivation and probable approaches to address such scenarios.

%\newpage
\section{Motivation}
\label{sec:intro:motiv}
One of the major challenges faced in stream mining is the lack of labeled data. There is no such problem in batched leaning. Data are finite and only, if any, an insignificant portion might be unlabeled. Streaming scenarios show a stark contrast. Most of the real world data are mostly unlabeled. It needs human intervention, resources, and time to properly label a stream data set. Most often only a fraction of data set is labeled by human expert or automated scripts are used. Owing to the fact of having such limitations, a large number of experimentations for stream mining algorithms are performed using synthesized data. It is much easier to control different parameters, concept drift, labeling, etc. in a generated data set. In most of the processes, this data generation process is mostly randomized, and probability of data generating from a certain region remains the same for entire hyperspace. Temporal locality are sometimes created by adding bias to certain region, however, the rates at which these regions produce data points mostly remain the same for all the regions in the hyperspace.

In reality many practical scenarios do not follow a uniform distribution for data generation. Some regions are more active than others. The term ``more active'' used here is in the sense that they produce more data. For example, Figure~\ref{fig:intro:nettraffic} shows the Internet traffic on an average day. A reference heat-map scale shows the blue color to be less than average and red to be higher than average. As the figure shows based on the time of the day, amount of network traffic in each region changes drastically even though the number of active nodes remains almost the same.

\begin{figure}[htbp] 
    \begin{center}
        \begin{tabular}{cc}
            \resizebox{70mm}{!}{\includegraphics{figs/nettraffic-a.png}} &
            \resizebox{70mm}{!}{\includegraphics{figs/nettraffic-b.png}} \\
            \scriptsize{(a)\vspace{2mm}} &
            \scriptsize{(b)}    
        \end{tabular}
        \caption{Worldwide Internet traffic. Day time in (a) western (b) eastern hemisphere \cite{internet:huffpost:nettraffic}}
        \label{fig:intro:nettraffic}
    \end{center}
\end{figure}

In another example, let's consider profiling of cell phone user based on the phone usages. Typically, young people are more inclined towards using data services than voice services while the professional and elderly people mostly rely on voice services. Thus, the profiling algorithm should consider this differences in rates in which different user groups are using the services.

To get a clearer picture of the locality of data activation and rate in which data are generated, let's consider the social media statistics. As of the second quarter of 2015, Facebook had 1.49 billion monthly active users. In the third quarter of 2012, the number of active Facebook users had surpassed 1 billion. Active users are those who log into Facebook during the last 30 days. It was only on the August 28, 2015 that Facebook had 1 billion user on a single day. Let assume that we want to do a sentiment analysis over the data set collected from a week of data of Facebook. Even though the class distribution (positive and negative sentiments)  may remain balanced, however, more active users will clearly overshadow the inputs given by the less active users. Similarly, about 300 million users produce 500 million tweets per day on Twitter. Figure~\ref{fig:intro:tweets} shows tweets for 5 different hashtags namely \#Trump, \#Syria, \#football, \#Volkswagen, and \#eclipse for the period of Aug 29, 2015 - Sept 28, 2015 (\cite{internet:topsy:tweets}). 
\begin{figure}[htbp] 
    \begin{center}
            \resizebox{110mm}{!}{\includegraphics{figs/tweets.pdf}}
        \caption{Number of tweets for different hashtags }
        \label{fig:intro:tweets}
    \end{center}
\end{figure}
\noindent As it can be seen, \#football has a weekly repetitive trend, most certainly because of weekend games. \#Trump (USA's presidential candidate for 2016), on the other hand, has a steady rate with occasional spikes depending on some highlighted events e.g. debate. Topic like \#Syria has lower rate, however, has a steady flow. Owing to the news of carbon emission from Volkswagen cars and lunar eclipse of September 28, 2015, topics like \#Volkswagen and \#eclipse got trending; which can be expected to disappear soon. If each of these topics is to be treated equally, slower streams like \#Syria would suffer from lack of data.

Based on the discussion above, it can evidently be seen that these streams do not necessarily follow a uniform distribution. Rather, these streams can be considered to be a collection of many sub-streams. Some properties of these sub-streams are follows:
\begin{itemize}
    \item A set of slow sub-streams generating low but relatively consistent number of data.
    \item A set of alternating sub-streams generating moderate number of data. Activation of sub-streams within the set is dependent to one other.
    \item A set of sub-streams that produces very large amount of data but only remain active for very short period of time.
\end{itemize}

Applying machine learning methods in such streams without considering the differences in the rate of data incoming for different sub-streams might lead to a set of decision rules dominated by the sub-streams with the highest number of instances. Moreover, most stream mining algorithms only keep track of most recent incoming instances and forget older data. Such algorithms would thus forget important decision rules learned over longer period of times for slow but consistent sub-streams when a heavy burst of occasional sub-streams arrives. It would take long time to learn the same rules again. Similar burst could lead to the deletion of decision rules for recurring sub-streams too. This could significantly affect the performance measures. Even for slow sub-streams overall performance measures may not reflect poor decision performances for those slow sub-stream, as the number of instances belonging to those streams are not high. But ideally it is expected that the mining algorithms should be equally effective for entire data space.

In this thesis, we address this scenario. To our best knowledge no previous work has specifically addressed this issue. In most works, evaluation are always performed with randomized streams. Concept drift, concept evolution, concept recurrence, etc. are addressed within the randomized streams. We extend one such data generation algorithm to implement the scenario presented above. Empirical evaluation are performed to compare the performances of existing algorithms for such streams. Comparing the results with the results from general randomized streams, it is found that existing algorithms do not perform at their best for streams with high temporal locality. Thus, this thesis presents an ensemble algorithm devised from one of the state-of-the-art algorithm to improve its performance. Extensive evaluation shows that new adaptation achieves more stable outcomes in overcoming the challenges posed by the nature of the stream. It also retains all the positive factors of the original algorithm.

Next section discusses the basic idea of the algorithm without going into mathematical details. We presented the algorithm later in great details once the related literatures and concepts are introduced in next couple of chapters.

\section{Intuition}
The central idea of our methods is based on the primal decision tree adaption by Domingos and Hulten~\cite{domingos00:vfdt} for streams named Very Fast Decision Tree (VFDT), also commonly known as Hoeffding Tree (HT). Their method is based on the assumption that to find the best attribute for a split decision at any node in a decision tree, it would be sufficient to consider only a certain amount of training data on that node. Hoeffding bound~\cite{hoeffding63:bound} is a measurement of degree of certainty for such approach, which gives an error bound for a decision taken after seeing a certain amount of instances. This bound essentially states that any decision taken after observing a certain amount of instances would remain the same after seeing an infinite number of instances and the error margin can be computed with this bound. Detailed discussion of this bound and the algorithm is presented in Chapter~\ref{chp:background}.

With such approach, challenges posed by high volume of data can be avoided. It does not require to remembering all the observed data instances. Rather, the statistics of the instances are sufficient to effectively decide about split attribute. One of the limitations of Hoeffding Tree is that it grows linearly as the time passes or instances arrives at the system. Root is the node that was splitted by observing the oldest set of examples. Similarly, decision rules at lower levels also become old, while the leaves contain the information from most recent instances. To keep the rules updated numerous approaches such as reseting the tree, pruning bad performing nodes, etc have been proposed. Detailed discussion of adaptations for concept drift, evolution, recurrence, etc. are being discussed in Chapter~\ref{chp:background}.

With reset and pruning facility Hoeffding Tree exhibits a special property. It always adapts itself for the newer examples. Number of examples HT's decision rules depend on is directly related to the size of the tree and number of examples required to split a node. Number of decision or split nodes could a measurement of the size of the tree. A smaller tree would adapt faster to the changes in the data. A larger tree would take longer time to adapt. Using this rationale a bagging method based of different sized trees is proposed by Bifet et al.~\cite{bifet09:asht}. In this method, a fixed number of Hoeffding Trees with different size (number of decision nodes) limits are used. Each time a tree exceeds the size threshold, the tree gets reset. Trivially, smaller trees would reset more often than the larger trees. And thus decision rules from smaller trees will base on the most recent data. Larger trees, however, would also have decision rules from older data.

We combined all these concepts to address our problem. First, we introduced a size restricted variant of Hoeffding Tree namely Size Restricted Hoeffding Tree (SRHT). Unlike its predecessor Adaptive Size Hoeffding Tree~\cite{bifet09:asht}, it does not get reset immediately after it reaches the size limit. Instead, it stops growing i.e. no further split occurs, however, the weights in the leaf nodes are updated with incoming instances. It also combines two different concepts introduces by its predecessors: (i) to reset once a size limit is reached and (ii) to maintain alternate trees or subtrees where necessary. Thus, even after reaching the size limit, a tree can switch to an alternate tree and start growing again till the limit is reached again. 

We use this setup for a modified bagging scheme introduced in~\cite{bifet09:asht}. Similar to the Hoeffding Tree, a smaller SRHT will have decision rules for most recent data, and a larger tree would contain rules for some older instances too. To put simply in our problem's context, smaller tree would have decisions for high speed sub-streams or burst of incoming streams. Whereas only the larger trees will have some decision rules for slow but consistent sub-streams, along with the decision rules for most recent data. This is because, the recent data are always dominated by high speed sub-streams. Smaller data do not get enough information about the slow stream or enough examples from slow streams to decide on them. Keeping this in mind, we control the reset of larger trees to keep hard-learned decision rules. In doing so, we maintain an alternate pool of trees that are to be reset soon, and start maintain a new tree with same size limit from scratch. For the transition time we consider votes from all the trees, thus effectively increasing the weights for slower streams. In Chapter~\ref{chp:algo}, we present more elaborate description of the algorithms.

\chapter{Related Works}
\label{chp:relworks}
Compared to data mining, stream mining is particularly a very young area. Most of the methods date back only couple of decades. Similarly concept of the ensemble learning was first introduced in the traditional batched learning, however, ensemble adaptations for streams are fairly newer concepts. In this chapter, we list the most influential methods developed so far with particular focus on tree based learners. We follow a rudimentary narrating style starting directly with methods for general stream learners for stationary streams, and then for evolving data. After that, we list the base methods for ensemble learning, and finally presents ensemble based stream learners.

\section{Stream Mining}
Compared to the classical data mining approaches stream mining is relatively a newer topic to be addressed in literature. Even though for batched approaches both classification and clustering problems have been vastly studied, their stream adaption remains a challenge due the restrictions imposed by the stream data. Possibility of temporal locality makes the classification problem harder in a streaming environment. Algorithms needs to address the evolution of underlying data stream. 

Domingos and Hulten introduced a strict one-pass adaptation of decision tree~\cite{breiman84:dt,quinlan93:c45} approach in streams. Classic approaches like ID3 and C4.5 learners assumes that all training examples can be stored in the main memory altogether. This is a significant limitation to the number of examples these algorithms can handle. Similarly, disk based decision tree learners (SLIQ~\cite{mehta96:sliq}, SPRINT~\cite{shafer96:sprint}, etc.) become very expensive when datasets are very large and the expected trees has many levels. Domingos and Hulten proposed \textit{Very Fast Decision Trees (VFDT)}~\cite{domingos00:vfdt} that uses Hoeffding bound~\cite{hoeffding63:bound} to build an anytime decision tree for constant memory and time. The primary assumption in this approach is that, to find the best attribute for a node in a decision tree, it may be sufficient to consider only a fraction of the training set that pass through that node. Hoeffding bound provides an statistical measure to determine how much data is needed to ensure a certain degree of certainty, i.e. error margin would be bounded by a given value~\cite{catlett91:thesis}.

In a recent approach, \cite{rutkowski13:vfdt} argued that using McDiarmid’s bound instead of Hoeffding bound in VFDT is more appropriate for ensuring the approximation bound i.e. the split decisions made after seeing certain number of instances will, with high probability, remain the same for infinite number of instances. The authors also presented McDiarmid’s bound for information gain of ID3 algorithm, and Gini index of CART algorithm. It is to be noted that Hoeffding bound is a special case of McDiarmid’s bound. In another paper, \cite{matuszyk:vfdt} showed usage of Hoeffding bound is mathematically flawed as (i) Hoeffding inequality only applies to arithmetic average, which information gain and Gini index are not, (ii) values obtained in sliding window methods are not independent, while Hoeffding inequality only applies to independent random variables. A revised bound showed that decision bound should be twice of the one given by Hoeffding bound, otherwise, error-likelihood should be updated accordingly.

Like most statistical and machine leaning algorithms VFDT assumes that training data is randomly drawn from a stationary distribution. This assumption is not valid for large databases and data streams. Over time underlying method or environment could change that generates data. The shift is sometimes also referred as {\it concept drift} in literature and can be abrupt as well as very slow. Data related to weather forecast, economic condition prediction, mis-calibrated sensors, etc. are examples of concept drifting environment. A concept-adaptive variant of VFDT, \textit{CVFDT}~\cite{hulten01:cvfdt}, can handle such scenarios. CVFDT updates its decision rules, essentially the tree structure, by detecting the concept drift in the data. It maintains alternate subtrees whenever an old subtree becomes questionable, and replaces the old one with the alternative when it become more accurate. CVFDT uses a sliding window and updates sufficient statistics by increasing the count of newly arrived examples and decreasing the count of old examples in the window. Essentially CVFDT achieves same accuracy that would be achieved if VFDT would have been run again with the new data. CVFDT does this in $O(1)$ with additional space requirement as compared to the VFDT's $O(w)$ where $w$ is the window size. A extension of VFDT, VFDTc was proposed in~\cite{gama05:vfdtc} that improves VFDT by adding continuous numeric attribute handling and naive Bayes prediction at the leaves.

In a different approach to handle concept drift Castillo et al.~\cite{gama03:drift} used Shewhart P-Charts in an online framework based on the idea of \textit{Statistical Quality Control}. Two alternatives of P-charts were used to monitor the stability of one or more quality characteristics in a drifting stream. The two alternatives only differed in the methods they estimate the target value to set the center. The group later introduced another drift detection scheme that monitors probability distribution of examples and maintains a online error rate to detect any concept drift~\cite{gama04:drift}. When distribution changes, error rate will increase. For stationary concept, the error rate should always gradually decrease. A new concept is said to be started if the error rate exceeds some predefined warning or threshold level. This approach has been used in \textit{Ultra Fast Forest Tree (UFFT)}~\cite{gama04:ft, gama05:ft} stream classification method. UFFT maintains naive Bayes statistics for every node. If at any node the error rate starts increasing, the node is pruned for drifting concept. UFFT uses similar approach and Hoeffding bound to control the growth of the tree. For each pair of classes a tree is maintained, hence it is called forest-of-trees.

Another decision tree based approach based on so-called \textit{Peano Count Tree (P-tree)} has been developed by Ding et al. for spatial data streams~\cite{ding02:peanocount}. The Peano Count Tree is a spatial data structure that facilitates a lossless compressed representation of spatial data. This structure is used for fast calculation of information gain for branching in decision trees.

Aggarwal et al. employs a slightly different idea in handling time-evolving data in their on demand classification approach of data streams ~\cite{aggarwal04:ondemand}. They used a modified \textit{micro-clusters concept} introduced in~\cite{aggarwal03:clustream}. Micro-clusters are created from the training data stream only. Each micro-cluster corresponds to a set of points from the training data belonging to the same class. To maintain statistics over different time horizons and avoid storage of unnecessary data points a geometric time frame is used. In the classification task, the \textit{$k$-nearest neighbor} based approach is taken, where micro-clusters are treated as node weighted by their instance counts. 

In~\cite{ganti02:gemm:focus} two algorithms named GEMM and FOCUS have been introduced for streams under block evolution. GEMM is used for model maintenance and FOCUS is for change detection between two data stream models. These algorithms has been tested using decision trees and frequent item set models. FOCUS uses bootstrapping methods to compute the distribution of deviation values when data characteristics remain the same. This distribution is then used to check whether the observed deviation value indicates a significant deviation. In another approach to handle concept drift, Last~\cite{last02:olin} proposed an online classification system OLIN that would dynamically adjust the size of the training window and the number of new examples between model re-constructions to the current rate of concept drift. OLIN uses constant resources to produce models, and achieves nearly the same accuracy as the ones that would be produced by periodically re-constructing the model from all accumulated instances.

Later, Aggarwal proposed an concept drift technique based on velocity density estimation \cite{aggarwal03:condrift}. Velocity density estimation is a technique to understand, visualize, and determine trends in the evolving data. The work presented a scheme to use velocity density estimation to create temporal velocity profiles and spatial velocity profiles at periodic instants in time. These profiles are then used to predict dissolution, coagulation, and shift in data. Proposed method could detect changes in trends in a single scan with linear order of number of data points. Additionally a batch processing techniques to identify combinations of dimensions which results the greatest amount of global evolution are also introduced. In~\cite{kifer04:condrift} authors tried to formally define and quantify the change so that existing algorithms can precisely specify when and how the underlying distribution has changed. They employed a two fixed-length window model, where a current one is updated every time a new example arrives and a reference one is only updated when a change has been detected. To compare the distributions of the windows $L1$ distance has been used [confirm!read again]. Another method to compare two distribution has been presented in~\cite{dasu05:condrift} where authors used Kullback-Leibler (KL) distance to compare two distributions. KL distance is known to be related to the optimal error in determining the similarity of two distributions. In this non-parametric method no assumptions on the underlying distributions is required. 

% TODO: briefly include multi-label evolving ...
% TODO: frequent item set mining]

\section{Ensemble Learning}
Traditional machine learning algorithms generally feature a single model or classifier such as \textit{Na\"ive Bayes} or \textit{Multilayer Perceptron (MLP)}. The free parameters of these learners (e.g. weights of feed-forward neural network) are set by realizing the complete training set. These classifier provides a measurement of the generalization performance i.e. how well the classifier generalizes the training set. However, given a finite set of training example, it is rather reasonable to assume that the data might contain several different generalization. For example, a different setting of neural network classifier (weights, node layers, node counts, etc.) changes the final network to some extent. For stream environment, this assumption becomes primitive. Thus, choosing a single classifier is not always optimal. Using the best classifier among several classifiers where each are trained with same training set would be an alternative, however, information is still being lost by discarding sub-optimal options. A better alternative would be to build a classifier ensemble. Ensemble classifiers combine the prediction of multiple base level model built on traditional algorithm. A simple process for combining prediction could be to choose the decision based on majority voting~\cite{parhami94:voting}. As demonstrated in several works~\cite{breiman93:regression, schapire90:whyens, wolpert92:whyens} ensemble methods (e.g. ensembles of neural networks)~\cite{hansen90:ensNN, tumer99:whyens} yield better performance. 

Without proper selection and control over the training process of the base learners, ensemble classifiers could result in poorer performance. Simply choosing a base classifier and training it for several settings would surely produce highly correlated classifiers which would have adverse effect on the ensemble process. One solution of this issue is to train each classifier with its own training set generated by sampling the original one. However, with random sampling each classifier would receive a reduced number of training patterns, resulting a reduction in the accuracy of the individual base classifier. This reduction in the base classifier accuracy is generally not recovered by the gain of combining the classifier unless measures are taken to make the base classifiers \textit{diverse}. Classifiers with complementary information would give the lowest correlation~\cite{breiman93:regression, tumer99:whyens}. Many methods have been proposed to promote diversity among the base classifier: \textit{bagging}~\cite{breiman94:bagging}, \textit{boosting}~\cite{drucker94:boosting, freund97:boosting, oza99:whyens}, \textit{cross-validation partitioning}~\cite{krogh95:ensNNcv, tumer99:whyens}, etc. These methods mainly process the entire training set repeatedly and require at least one pass for each base model. This is not suitable for streaming scenarios. Stream adaptation of bagging and boosting methods has been introduced by Oza et al.~\cite{oza01:obagboost,oza01:thesis}.

\subsection{Ensemble Learning in Streams}
Learning algorithms in data streams require maintenance of a hypothesis based on the training instances seen thus far with the need for storage and reprocessing. Facilitating this requirement, Oza and Russell developed an online version~\cite{oza01:obagboost, oza01:thesis} of traditional bagging and boosting. Bagging works by randomly sampling with replacement from the training set to form a given number of intermediate training sets which are used to train same number of classifiers. During testing a majority voting scheme is employed on the decisions of all classifiers to deduce the final decision. Boosting uses an iterative procedure to adaptively change distribution of training data by focusing more precisely on misclassified instances. Initially all instances have equal weights, and at the end of a boosting round weight of each instance is updated by increasing or decreasing if the instance was classified wrongly or correctly, respectively. For \textit{online} variant of these algorithms, not knowing the size of the training data poses a problem in determining the size of training sets to build the base models. In~\cite{oza01:obagboost} authors address this situation by training $k$ models with each instances where $k$ is a suitable Poisson random variable. Later on,~\cite{pelossof08:boosting} proposed the \textit{Online Coordinate Boosting} algorithm where the number of weight updates of~\cite{oza01:obagboost} is reduced using few simple alteration.

Online bagging and boosting method do not particularly give attention to the concept drifting nature in the data. \textit{Accuracy Weighted Ensemble (AWE)}~\cite{wang03:accuweighted} is one of the earliest work on concept-drifting stream data. AWE assumes that the stream is delivered in chunks of defined size. With each incoming chunk, AWE updates its $k$ classifiers. Each classifier is associated a weight which is inversely proportional to the expected error of the respective classifier. To estimate this error, it is assumed that the distribution of test set is closest to the most recent chunks. Concept drift is adapted by effectively manipulating the number and the magnitude of the weights that are changing. An extension of AWE has been proposed in~\cite{brzezinski11:accuupdated}, namely \textit{Accuracy Updated Ensemble (AUE)}. AUE takes the weighting motivation from AWE, but improves the limitation of AWE. In AWE each classifier learns from the incoming chunks in a ``batched'' fashion. AUE employs an online scheme instead. AUE also adapts the weighting function to reduce the adverse effect in AWE of sudden drift in data. AWE weighting function is prone to suffer by rapid change in the stream and most or even all classifiers assuming they are ``risky''. This limitation has been addressed in AUE. Result shows that AUE performs marginally better than AWE, however, also requires slightly longer time and larger space.

As mentioned in the previous section, \textit{Hoeffding Tree (HT)} i.e. VFDT~\cite{domingos00:vfdt}, can be used to build classifiers for concept drifting streams. The Hoeffding Tree has the property that it adapts itself for the newer examples. The number of examples that a HT is build upon is determined by two numbers: (i) the be size of the tree, and (ii) the number of examples used to create a node. Thus, smaller trees adapt faster to the changes in the data, while larger trees tries to retain the rules that reflect longer time-frame, simply because they are built on more data. In other words tree bounded by size n would be reset twice as often as tree bounded by size $2n$. \textit{Adaptive Size Hoeffding Tree (ASHT)}~\cite{bifet09:asht} uses this intuition to build an ensemble of classifiers of different sized Hoeffding trees. ASHT attempts to increase the diversity in the bagging approach. The maximum allowed size for $n$-th tree is twice the size of $(n-1)$-th, where the 1st tree has a size of $2$. Additionally inverse of the squared error has been used as the weights for the trees. Diversity between the traditional bagging and ASHD bagging are compared using kappa statistic. If two classifier agree on every example then $k = 0$, and if they agree on the predictions purely by chance then $k = 0$. To test this approach authors have used Interleaved Test-then-Train method. At the leaf level of HT, Naive Bayes predictions are used. The experiments were performed on several different generated dataset such as SEA Concepts Generator, STAGGER Concepts Generator, Rotating Hyperplane, Random RBF Generator, etc. Performance are compared with traditional Naive Bayes, HT, and boosting methods. Evaluation concluded that bagging provides best accuracy, however, with the higher cost in terms of running time and memory. Authors made an observation that even bagging using 5 trees of different size might be sufficient for gain higher accuracy, as error level for bagging with 10 trees does not drop much but takes twice time.

Same authors also proposed an adaptive window size bagging method- \textit{ADWIN}~\cite{bifet09:asht}. ADWIN automatically detects and adapts to the current rate of change. To do so ADWIN adapts its window size to maximize the statistically consistently length that conforms following hypothesis ``there has been no change in the average value inside the window''. Window is not maintained explicitly rather using a variant of exponential histogram technique that takes $O(log w)$ memory and $O(log W)$ processing time where $w$ is the length of the window. Experimental evaluation showed that ADWIN has better accuracy than ASHT, however, requires more time and memory. 

ADWIN has later been used in \textit{leverage bagging}~\cite{bifet10:levbag}. Leverage bagging improves randomization by increasing resampling and using output detection codes. Resampling with replacement is done in Online Bagging using Poisson(1). Instead leverage bagging increases the weights of resampling using a larger value $\lambda$ to compute the value of the Poisson distribution. The Poisson distribution is used to model the number of events occurring within a given time interval. In other improvement randomization is added at the output of the ensemble using output codes. Method works by assigning a binary string of length $n$ to each class and building an ensemble of $n$ binary classifiers. Each of the classifiers learns one bit for each position in the string. A new instance is classified to the class whose binary code is closest. 

ADWIN has also been used in building ensemble of \textit{Restricted Hoeffding Trees}~\cite{bifet10:rht}. A mechanism for setting the learning rate of perceptrons using ADWIN's change detection method is used to restrict the tree. Additionally, a mechanism for reseting the member Hoeffding trees is also been introduced when a particular member is no longer performing well. The method outperforms traditional bagging in terms of accuracy, but requires additional memory and time.

Another group of methods known as \textit{random forests} uses bagging of decision trees. The concept of random forest is introduced by Breiman~\cite{breiman99:randomforest}. Random forest works in a similar fashion as bootstrap aggregation. However, for each split in the tree it only selects a subset of the features to be considered for splitting criterion. This \textit{feature bagging} approach helps avoiding very strong predictors to get selected over and over. One method mentioned before, \textit{Ultra Fast Forest Tree (UFFT)}~\cite{gama04:ft, gama05:ft} uses concepts of tree bagging, however, works with all features the whole time. UFFT maintains statistical information on each node to detect drift and grows the tree in a similar  approach to VFDT. Using the statistical information stored in nodes it detects concept drift with naive Bayes error-rate. Abdulsalam et al. proposed \textit{Dynamic Streaming Random Forests (DSRF)} focusing on lowering the number of examples required to build up new model in a drifting environment~\cite{salam08:dsrf, salam11:dsrf}. Shannon entropy~\cite{shannon01:entropy} is used to detect concept drift. All these methods are empirically proven to be effective for generalized scenario and chosen data sets. 



\section{Thesis Accomplishments}
The primary objective of this thesis is to devise an algorithm realizing the intuitions discussed earlier. In doing so, an extensive survey of existing literatures is performed. The most relevant portion of which is already been presented in previous section. The rest is attached at the end of this thesis (see Appendix~\ref{appndx:erw}). This thesis presents a new way to look at the composition of various real-world streams e.g. social networks. Based on that, a new data generation of approach to facilitate slow, fast, burst, recurrent, etc. behavior in a randomized data stream is proposed. In an effort to combine the ideas of adaptive size Hoeffding tree and adaptive Hoeffding tree using ADWIN, a new ensemble method is devised aiming to improve performance for slower streams. Lastly, a comprehensive empirical comparison of current Hoeffding tree based approaches is performed and justification of the new approach is performed using synthesized data.


\section{Outline}
The rest of the article is organized as follows: Chapter~\ref{chp:background} presents the base concepts required for the algorithm. We first discuss the basic streaming adaptations of batched methods, and later their usages in ensemble approaches. Chapter~\ref{chp:algo} precisely defines the problem and discusses the development of our algorithm addressing the motivation and intuition of this chapter. In Chapter~\ref{chp:dataset} existing data set generators are discussed first and a new data generation scheme is explained that fulfills the requirements discussed in the intuition section. Finally, before concluding the thesis in Chapter~\ref{chp:conclude}, in Chapter~\ref{chp:exp} we detail the experimental evaluation process and findings. Extensive evaluation is performed to compare existing methods, as well as, method devised in this thesis  (Chapter~\ref{chp:algo}). 
